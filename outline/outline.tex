\documentclass[11pt]{article}
\usepackage[margin=1in]{geometry}
\usepackage{amsmath,amssymb,amsthm}
\usepackage{outlines}
\usepackage{enumitem}

%formatting the different levels of the outline to be I. -> A. -> i. -> a.
\setenumerate[1]{label=\Roman*.}
\setenumerate[2]{label=\Alph*.}
\setenumerate[3]{label=\roman*.}
\setenumerate[4]{label=\alph*.}




\title{Fermat Outline}
\author{Alex Bechanko}

\begin{document}
\maketitle
\begin{outline}[enumerate]
\1 Exploration of Perfect Numbers and Primality
    \2 Perfect Numbers
        \3 Definition
            \4 A positive integer that is equal to the sum of its proper divisors.
            \4 A positive integer that is equal to half the sum of its positive divisors.
        \3 Results found by other mathematicians that Fermat based his propositions off of
            \4 Book 4 of Euclid's Elements: if $2^n-1$ is prime, then $2^{n-1}(2^n-1)$ is perfect
        \3 Fermat's 3 propositions regarding perfect numbers
            \4 Given an integer $n$, if $n$ is not prime then $2^n - 1$ is not prime. Show why this must be using the factoring method.
            \4 Given an integer $p$, if $p$ is an odd prime then $2p$ is a divisor of $2^p-2$
            \4 Given an integer $p$, if $p$ is an odd prime then the only divisors of $2^p-1$ are of the form $2pk+1$ for
               some integer $k$
    \2 Mersenne Primes
        \3 Definition
            \4 A prime number of the form $2^n - 1$, for some integer $n$.
        \3 Results found by Fermat
            \4 The first of the three propositions that Fermat sent to Mersenne
        \3 Created by Fermat, but were called Mersenne primes by others due to existence of Fermat numbers
    \2 Fermat's Little Theorem
        \3 Statement of theorem
            \4 For any integer $a$ and prime $p$, $a^p - p$ is an integer multiple of $p$.
            \4 It's more succinctly known as \[a^p \equiv a \pmod{p}\]
        \3 This was a generalization of the last two propositions sent to Mersenne
            \4 For the most part, replace 2 with an arbitrary integer $a$ and you have the generalization
            \4 The other parts are corollaries to this result
    \2 Fermat Numbers
        \3 Definition
            \4 Given an integer $n$, a Fermat number is of the form $2^{2^n}+1$
        \3 Fermat's conjectures about Fermat Numbers
            \4 All of them are prime numbers, and he wrote that he had a proof of this result
        \3 A few examples that seem to works
            \4 for $n=0,1,2,3,4$ Fermat numbers are prime
        \3 An counter-example to his conjecture
            \4 Euler's discovery of a Fermat number that was not prime
            \4 Euler found divisors for $n=5$
        \3 Error in his judgements
            \4 It is conjectured that his proof was done using a proof by infinite descent, a proof similar to that of an
               inductive proof.
            \4 He may have showed it to work for all number below some given $n$, and did not check for $n$ greater then $4$.
            \4 He was also notoriously bad at structuring his results for publishing. This was mostly due to his use of
               a notation that nobody used anymore.

\1 Rediscovery of Proof of Infinite Descent
    \2 Description of how it works
        \3 Assume that a result works for some arbitrary positive integer $n$.
        \3 Infer that $n-1$ works if the given $n$ works.
        \3 Continuing this argument, you get an infinite sequence of decreasing positive numbers, which is not possible.
        \3 Thus the result can not be true for positive integers.
    \2 The only example of Fermat's use of proof by infinite descent
        \3 If a triangle has integral sides, then its area is not integral.
        \3 It is impossible to find integers $x,y,z,w$ so that $x^2 + y^2 = z^2$ and $\frac{xy}{2} = w$.
        \3 show proof
\1 Fermat's Last Theorem
    \2 Statement of theorem
        \3 There are no three integers $a,b,c$ so that $a^n + b^n = c^n$ for $n>2$
    \2 First conjectured by Fermat in the margins of his copy of \emph{Arithmetica}
    \2 Was supposedly proven as well, but Fermat did not have enough room in the margins to show it.
    \2 No proof found until 1995 despite popularity
    \2 The mathematics required in the proof implies that Fermat's proof was probably wrong
\1 Primality tests that use results by Fermat
    \2 Fermat Primality Test
        \3 uses Fermat's little theorem to help determine if a number is composite
        \3 Basic algorithm
            \4 To determine if $p$ is prime, pick any $a$ so that $0 < a < p$.
               If $a^p \not\equiv a \pmod{p}$ then $p$ must be composite.
               Repeat this algorithm for more integers $a$ to be even more sure that $p$ is prime
        \3 Flaws
            \4 Can't determine if a number is prime, only that it is very likely to be prime.
            \4 There are composite numbers that satisfy Fermat's little theorem, that this algorithm does not account for.
        \3 similar and more useful test is Miller-Rabin primality test
            \4 Uses Fermat's Little Theorem as well
            \4 There is a version that proves a number is prime, but its based off the unproven Riemann hypothesis
            \4 The version not based on Riemann hypothesis uses probabilistic methods for determining primality.
    \2 Lucas-Lehmer
        \3 A primality test for Mersenne numbers
        \3 Uses a stronger form of Fermat's little theorem called Lehmer's theorem to determine primality
        \3 Basic Algorithm
            \4 Computes a number called the Lucas-Lehmer residue for the Mersenne number.
                If the residue is 0, then the number is prime, otherwise its composite.
            \4 Define the number $s_i, i \ge 0$ where
                \[s_i = \begin{cases}
                    4,           & \text{if } i=0\\
                    s^2_{i-1}-2, & \text{otherwise}
                \end{cases}\]
            \4 There is a theorem that shows \[s_{p-2}\equiv0 \pmod{2^{p}-1}\] if and only if $2^p-1$ is prime.
            \4 This algorithm is based around computing $s_i$
        \3 Still currently used for finding large primes
        \3 This algorithm proves that the number is prime, unlike Fermat primality test
\end{outline}

\begin{thebibliography}{9}
\bibitem{Ball}
    Ball, W. W. Rouse. \emph{A Short Account of the History of Mathematics}. New York: Dover, 1960. Print.

\bibitem{Mahoney}
    Mahoney, Michael S. \emph{The Mathematical Career of Pierre De Fermat, 1601-1665}. Princeton, NJ: Princeton UP, 1994. Print.

\bibitem{Wirth}
    Wirth, Claus-Peter.
    \emph{A Self-Contained and Easily Accessible Discussion of the Method of Descente Infinie and Fermat's Only Explicitly Known Proof by Descente Infinie}.
    Seki Working-Paper (2006): n. pag. Web.

\bibitem{Kleiner}
    Kleiner, Israel.
    \emph{From Fermat to Wiles: Fermat's Last Theorem Becomes a Theorem}.
    Elemente Der Mathematik 55.1 (2000): 19-37. Print.

\bibitem{Euclid}
    Euclid. Euclid's Elements;. N.p.: Dutton, 1933. Print.

\bibitem{Miller}
    Miller, Gary L.
    \emph{Riemann's Hypothesis and Tests for Primality}.
    Journal of Computer and System Sciences 13.3 (1976): 300-17. Print.

\bibitem{MersennePrimeTests}
    Jason Wojciechowski.
    \emph{Mersenne Primes, An Introduction and Overview}. 2003.

%\bibitem{wFermat}
%    \emph{Pierre De Fermat}. Wikipedia. Wikipedia, n.d. Web. $<$wikipedia.org$>$.

\bibitem{wPerfectNumbers}
    \emph{Perfect Numbers}. Wikipedia. Wikipedia, n.d. Web. $<$wikipedia.org$>$.

%\bibitem{wFermatsLittleThereom}
%    \emph{Fermat's Little Theorem}.  Wikipedia. Wikipedia, n.d. Web. $<$wikipedia.org$>$.

%\bibitem{wFermatsLastTheorem}
%    \emph{Fermat's Last Theorem}.  Wikipedia. Wikipedia, n.d. Web. $<$wikipedia.org$>$.

\bibitem{wFermatTest}
    \emph{Fermat Primality Test}. Wikipedia. Wikipedia, n.d. Web. $<$wikipedia.org$>$.

%\bibitem{wLucasLehmerTest}
%    \emph{Lucas-Lehmer Primality Test}.  Wikipedia. Wikipedia, n.d. Web. $<$wikipedia.org$>$.

%\bibitem{wMillerRabinTest}
%    \emph{Miller-Rabin Primality Test}.  Wikipedia. Wikipedia, n.d. Web. $<$wikipedia.org$>$.


\end{thebibliography}
\end{document}