\documentclass[11pt]{article}
\usepackage[margin=1in]{geometry}
\usepackage{amsmath,amssymb,amsthm}
\usepackage{graphicx}

\title{Fermat's Work and Contributions to Number Theory}
\author{Alex Bechanko}
\begin{document}
\maketitle
There are many things that Fermat is famous for today, but by far his most widely
known research was in Number Theory, namely Fermat's Little Theorem, and 
Fermat's Last Theorem.
This paper will discuss some of his lesser
known contributions to Number Theory, which includes the exploration of
Perfect numbers, as well as the primality of numbers.
We will also go over one of his more commons methods for proving general results,
the proof by infinite descent.
His little theorem will also be thoroughly explained, including several 
interesting primality results that require his theorem.
There will also be discussion regarding how Fermat's work in number
theory was regarded to other mathematicians at the same time period, as
well as further on into the future. 

\section*{Introduction}
It is widely regarded that Fermat created modern number theory.
Most mathematicians, including Fermat, at Fermat's time were focused
on many topics in mathematics such as slopes of curves and areas under curves,
there were not many if at all that chose to take a step back, and look at really
old results with the the lens of the new results to focus in on different areas
of the old results \cite{Mahoney}. 
Despite what results Fermat obtained, it's interesting to note that there were
not many people that were interested in his work related to number theory.
Any efforts that Fermat made to make his number theory results known were fairly
useless, as most of his correspondents were not interested in this new field of
mathematics he was creating.
As a result, most of his methods for number theory were very difficult or
impossible to obtain
for this day and age by historians, because after a he only outlined
his results for these topics to the many correspondents that he wrote
to.
One of the only other places historians were able to find results of his was in
a copy of one of his mathematics books \emph{Arithmetica} written by Diophantus.
In fact, even once interest was sparked in his results a century or so later,
only his results were able to be founded, and not his results.
So later mathematicians found it much easier to create their own methods
instead of recreating his \cite{Mahoney}.

\section*{Exploration of Perfect Numbers and Primality}

\subsection*{Perfect Numbers}
One of Fermat's first correspondents was the Marin Mersenne, who was said to be
the center of of mathematics in the 16th century \cite{Mersenne}.
Mersenne had heard of Fermat even before this correspondence, in particular it
was Fermat's observations of the divisibility of numbers that had garnered 
Mersenne's attention.
Including this was perfect numbers, and their relations to prime
numbers \cite{Mahoney}.
Perfect numbers are numbers whose proper positive divisors add up to
themselves \cite{PerfectNumbers}.
Equivalently, you could say that a number is perfect if its positive divisors
add up to twice the number.
A simple example of a perfect number would be 6, whose proper divisors are
1,2, and 3, which sum to 6.

There were a few reasons for mathematicians interest in perfect numbers.
The first reason was that while there was a way to generate perfect numbers,
the way to generate them itself was a very difficult process.
In particular, in Euclid's \emph{Elements} has the following result.
If $2^{n+1} + 1$ is prime, then \[ 2^n(2^{n+1} + 1)\]
is a perfect number \cite{Euclid,Mahoney}.
So the perfect number problem had transformed into methods for
finding prime numbers.

\subsection*{Mersenne Primes}
It is through the emphasis on the topic of primality that Fermat had garnered the
attention of a less well-known number theorist named
Bernard Frenicle de Bessy \cite{Mahoney}.
It was previously said that Fermat tended to hold his methods for finding many
of his results to himself, but thanks to the correspondence with Bernard we
know some of the methods Fermat used to determine primality of numbers.
The first proposition he gave was this.
Given a positive integer $n$, if $n$ is composite, then $2^n - 1$ is also 
composite.
Indeed this can be shown with a direct factoring.
Let $n = ab$, then
\[2^{ab} - 1 = (2^a - 1)(1 + 2^a + 2^{2a} + \dots +2^{(b-1)a} ). \]
The next proposition that he showed in his correspondence was that given a 
positive integer $p$, if $p$ is an odd prime then $2p$ divides $2^p-2$.
Put in a simpler way, if $p$ is an odd prime then $p$ divides $2^{p-1} - 1$.
Indeed, this is an application of Fermat's little theorem, however it is thought
that Fermat had not yet developed the general theorem \cite{Ball, Mahoney}.
Finally, Fermat also stated in his correspondence with Bernard that given an
integer $p$, if $p$ is an odd prime then the only divisors of $2^p -1$ are of
the form $2pk +1$ for some integer $k$.
Again, this was another application of Fermat's little theorem, but to get a
useful list of numbers to check divisibility with \cite{MersennePrime}.
Later on, mathematicians and historians alike would come to know prime numbers
of the form $2^p -1$ as Mersenne primes, despite the fact that it was Fermat
that first brought light onto these type of numbers \cite{Mahoney}.
Admittedly, the main reason for this was due to Fermat naming a different set
of numbers after himself.
It was thanks to Marin Marsenne for computing some of the Mersenne primes that
gave these prime numbers his last name.

\subsection*{Fermat's Little Theorem}
Later on in Fermat's correspondence with Bernard and Mersenne, he finally divulged
that he had a general theorem with which he was deriving some of his results.
Although, the form is a bit different to what we see today \cite{Mahoney}: 
    \begin{quote}
        Without exception, every prime number measures one of the powers $- 1$    
        of any progression whatever, and the exponent of the said power is a
        submultiple of the given prime number $-1$.
        Also, after one has found the first power that satisfies the problem,
        all those which the exponents are multiples of the exponent of the first
        will satisfy the problem.
    \end{quote}
Perhaps it is a little cryptic, but in a more succinct form, he is saying that
given a prime $p$, and a sequence of $a^t - 1$ where $a$ and $t$ are positive
integers, then $p$ divides at least one member of that sequence, and $t$ divides
$p-1$.
He then goes on to say that $p$ will divide all members of the sequence of the
form $a^{tk} - 1$ for all positive integers $k$.
In modern notation, if $p$ is prime and any integer $a$,
\[a^p \equiv a \pmod{p}. \]
Fermat only spoke of this theorem in this letter to Bernard, and probably due to
the shift in interests towards Pythagorean triples, there was no one pressing
to have Fermat give his proof, or at least an outline of the proof.

\subsection*{Fermat Numbers}
As discussed earlier, despite having been the first documented to research 
Mersenne Primes, these numbers were not given Fermat's name due to the existence
of a different set of numbers that Fermat named after himself.
A Fermat number is a number of the form $2^{2^n} +1$.
This number arose from his work with Mersenne numbers, and indeed as he
researched them he conjectured that all Fermat numbers are prime numbers.
Indeed, because of how quickly Fermat numbers grow in size he could only
test for cases $0,1,2,3,4$, which are
    \[\{3, 5, 17, 257, 65537\}. \]
What's interesting about this conjecture is that this is one of the few Fermat
was resolute on it being true, and had tried many times to prove this result.
It was only later in the late 17th century that Leonhard Euler showed that the
this conjecture was not true by giving a factorization of the 5th case,
    \[2^{2^5}+1 = 4294967297 = 641 \cdot 6700417. \]
Fermat even stated in multiple letters to Mersenne of his confidence of this
conjecture, as well as the frustration with finding a proof for
this result \cite{Mahoney, Ball}.
It is conjectured by a few historians/mathematicians that Fermat had used some
form of a proof by infinite descent to help prove his conjecture, but there is
nothing stated in the letters about this to Bernard or Mersenne.
Another possibility on how he got this result wrong was that he was notoriously
bad at structuring his results for publishing, so it is conjectured
that he made some errors in his process here as well.

\section*{Rediscovery of the Proof \emph{Descente Infinie}}
One of the peculiarities of Fermat was that he was one of the few of his time
to regularly use the proofing method called proof \emph{descente infinie}, or
proof by infinite descent.
The proof by infinite descent is a proof method that aims to prove the negation
of a claim.
To do this you would assume that the opposite holds for some arbitrary
positive integer value.
If you could infer from this value that a smaller value existed, then you could
reach the conclusion that there are infinitely many such smaller values.
However, this would be a contradiction, because there are not an infinite amount
of positive integers below any specified positive integer, and thus the claim
would be not true.

We first learn of Fermat's use of this method of proving from his correspondence
with a man named Christiaan Huygens \cite{Mahoney, Wirth}.
Christiaan was among other things a well known Dutch mathematician.
One of his works that he is known for was the creation of the first book
on probability theory called \emph{De ratiociniis in ludo aleae} \cite{Huygens}.
Fermat wrote an example of this method to Christiaan in response to Christiaan's
curiosities of his methods \cite{Mahoney}:

\begin{quotation}
    If there were any right triangle that had an area equal to a square,
    then there would be another triangle less than that one which would
    have the same property.
    If there were a second less than the first which had the same property,
    there would by similar reasoning be a third less than the second
    which would have the same property, and then a fourth, a fifth, etc.,
    descending \emph{ad infinitum}.
    Now it is the case that, given a number, there are not infinitely many
    numbers less than that one in descending order (I mean always to speak in
    integers). 
    Whence one concludes that there is a square \ldots

    I do not add the reasoning by which I infer that, if there was a right
    triangle of that nature, there would be another of the same nature less
    than the first, because the argument would be too long and because that is
    the whole mystery of my method.
    I will be content if the Pascals and Robervals and so many learned men
    search for it according to my indications.
\end{quotation}

So as per Fermat's way of speaking with his correspondents, he left showing that
the method worked as an exercise for them to work out.

Later on closer to modern day, historians would find Fermat's copy of 
Diophantus' \emph{Arithmetica}, and in it they would find one of the few proofs
that they would ever find written by Fermat \cite{Wirth}.
The result that he proved with this method was that if a triangle has integer
side lengths, then the area of the triangle can not be represented by a square
with integer side length.
More succinctly, if $x_0,x_1 \in \mathbf{N}$ and $x_2,x_3 \in \mathbf{N}$ and
and $x_0^2 + x_1^2 = x_2^2$, then \[\frac{x_0x_1}{2} \neq x_3^2. \]
The way Fermat writes this proof is similar to how he writes to his correspondents,
that is he leaves many parts left to the reader to prove if they do not believe
Fermat's assertions \cite{Wirth}.



\section*{Results Based off Fermat's Work}

\begin{thebibliography}{9}
\bibitem{Ball}
    Ball, W. W. Rouse. \emph{A Short Account of the History of Mathematics}. New York: Dover, 1960. Print.

\bibitem{Mahoney}
    Mahoney, Michael S. \emph{The Mathematical Career of Pierre De Fermat, 1601-1665}. Princeton, NJ: Princeton UP, 1994. Print.

\bibitem{Wirth}
    Wirth, Claus-Peter.
    \emph{A Self-Contained and Easily Accessible Discussion of the Method of Descente Infinie and Fermat's Only Explicitly Known Proof by Descente Infinie}.
    Seki Working-Paper (2006): n.\ pag. Web.

\bibitem{Kleiner}
    Kleiner, Israel.
    \emph{From Fermat to Wiles: Fermat's Last Theorem Becomes a Theorem}.
    Elemente Der Mathematik 55.1 (2000): 19-37. Print.

\bibitem{Euclid}
    Euclid. Euclid's Elements;. N.p.: Dutton, 1933. Print.

\bibitem{Miller}
    Miller, Gary L.
    \emph{Riemann's Hypothesis and Tests for Primality}.
    Journal of Computer and System Sciences 13.3 (1976): 300-17. Print.

\bibitem{MersennePrimeTests}
    Jason Wojciechowski.
    \emph{Mersenne Primes, An Introduction and Overview}. 2003.

%\bibitem{wFermat}
%    \emph{Pierre De Fermat}. Wikipedia. Wikipedia, n.d. Web. $<$wikipedia.org$>$.

\bibitem{PerfectNumbers}
    \emph{Perfect Numbers}. Wikipedia. Wikipedia, n.d. Web. $<$wikipedia.org$>$.

\bibitem{Mersenne}
    Bernstein, Peter L. (1996).
    \emph{Against the Gods: The Remarkable Story of Risk}. John Wiley \& Sons.\ p. 59.
%nbibitem{wFermatsLittleThereom}
%    \emph{Fermat's Little Theorem}.  Wikipedia. Wikipedia, n.d. Web. $<$wikipedia.org$>$.

%\bibitem{wFermatsLastTheorem}
%    \emph{Fermat's Last Theorem}.  Wikipedia. Wikipedia, n.d. Web. $<$wikipedia.org$>$.

\bibitem{FermatTest}
    \emph{Fermat Primality Test}. Wikipedia. Wikipedia, n.d. Web. $<$wikipedia.org$>$.

%\bibitem{wLucasLehmerTest}
%    \emph{Lucas-Lehmer Primality Test}.  Wikipedia. Wikipedia, n.d. Web. $<$wikipedia.org$>$.

%\bibitem{wMillerRabinTest}
%    \emph{Miller-Rabin Primality Test}.  Wikipedia. Wikipedia, n.d. Web. $<$wikipedia.org$>$.

\bibitem{MersennePrime}
    \emph{Mersenne Primes}. Wikipedia. Wikipedia, n.d. Web. $<$wikipedia.org$>$.

\bibitem{Huygens}
    \emph{Christiaan Huygens}. Wikipedia. Wikipedia, n.d. Web. $<$wikipedia.org$>$.

\end{thebibliography}

\end{document}

